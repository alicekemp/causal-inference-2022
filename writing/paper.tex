\documentclass{article}

% these packages let you do math
\usepackage{amsmath}
\usepackage{amssymb}

% we need these packages for fancy R tables
\usepackage{booktabs}
\usepackage{float}
\usepackage{colortbl}
\usepackage{xcolor}

% these packages play with the spacing/margins of the document. Uncomment the commands on lines 16 and 17 to see what they do.
\usepackage{a4wide}
\usepackage{setspace}
\usepackage{geometry}
\usepackage{parskip}
%\doublespacing
%\geometry{margin=1.5in}

% this package helps us with including images. Setting the graphics path makes it easier to refer to things in the \includegraphics command.
\usepackage{graphicx}
\graphicspath{ {../figures/} }

% make some hyperlinks using the \href command
\usepackage{hyperref}
\hypersetup{
    colorlinks=true,
    linkcolor=black,
    urlcolor=blue
}

% set the author, title, and date of the document. \maketitle adds it to the document.
\author{Nathan Hattersley}
\title{My Paper on NLSY97 Data}
\date{Sping 2022}

\begin{document}
\maketitle

\section{The First Section}

This is where I talk about basic \LaTeX. Using the \texttt{parskip} package, I can create a new paragraph by using line breaks, which I will do now.

I can also make cool equations inline by using parentheses{\textemdash}like this: \(x + 2\){\textemdash}or by using single dollar signs{\textemdash}like this: $x+2$. Parentheses are preferred because the left and right delimiter are distinct.

I can make cool equations in a block style by using the \texttt{equation} environment like so:
\begin{equation*}
    y = x\beta + \varepsilon
\end{equation*}
or by using double dollar signs:
$$
    y = x\beta + \varepsilon
$$

Again, the \texttt{equation} environment is preferred because the begin and end delimiters are different.

I can also add a bibliography, but this is beyond the scope of our discussion right now. Overleaf has plenty of resources for this on their \href{https://www.overleaf.com/learn}{website}. Another good place to look for LaTeX help is the \href{https://en.wikibooks.org/wiki/LaTeX}{WikiBook} on it.

\newpage

\section{The Second Section}

Wherein we do tables and graphs. To include the graph we made in ggplot, we create the \texttt{figure} environment. The `H' option tells LaTeX to `hold' the position of the figure instead of positioning it somewhere else. I use the \texttt{caption} command to add a caption{\textemdash}although I also put a title on the plot in ggplot so you would typically choose one or the other. I use the \texttt{label} command after the caption to add a label. Then in my paper I can use the \texttt{ref} command and LaTeX knows I am referring to Figure \ref{fig:graph}.


\begin{figure}[H]
    \begin{center}
        \includegraphics[width=.85\textwidth]{arrests_by_racegender}
    \end{center}
    \caption{Mean Number of Arrests in 2002 by Race and Gender (this is the LaTeX caption, not the ggplot title)}
    \label{fig:graph}
\end{figure}


Tables are somewhat easier, since \texttt{kableExtra} and \texttt{stargazer} generate LaTeX code that is ready to just ``copy-paste'' into our document. The \texttt{label} argument in the R code is the label that the table will have in the tex output, if you want to \texttt{ref} it.

\begin{table}[H]

\caption{\label{tab:tab:summarystats}Mean arrests in 2002 by Race and Gender}
\centering
\begin{tabular}[t]{lrrrr}
\toprule
Gender & Black & Hispanic & Mixed Race Non Hispanic & Non Black Non Hispanic\\
\midrule
\cellcolor{gray!6}{Female} & \cellcolor{gray!6}{0.0211268} & \cellcolor{gray!6}{0.0298013} & \cellcolor{gray!6}{0.1428571} & \cellcolor{gray!6}{0.0193192}\\
Male & 0.4876712 & 0.1579509 & 0.0000000 & 0.1099476\\
\bottomrule
\end{tabular}
\end{table}



% Table created by stargazer v.5.2.2 by Marek Hlavac, Harvard University. E-mail: hlavac at fas.harvard.edu
% Date and time: Thu, Jan 13, 2022 - 12:45:58
\begin{table}[!htbp] \centering 
  \caption{Regression Output. Omitted category is Black Females.} 
  \label{tab:regression} 
\begin{tabular}{@{\extracolsep{5pt}}lc} 
\\[-1.8ex]\hline 
\hline \\[-1.8ex] 
 & \multicolumn{1}{c}{\textit{Dependent variable:}} \\ 
\cline{2-2} 
\\[-1.8ex] & Arrests in 2002 \\ 
\hline \\[-1.8ex] 
 Hispanic & $-$0.055$^{***}$ \\ 
  & (0.019) \\ 
  & \\ 
 Mixed Race (Non-Hispanic) & $-$0.084$^{*}$ \\ 
  & (0.043) \\ 
  & \\ 
 Non-Black / Non-Hispanic & $-$0.056$^{***}$ \\ 
  & (0.017) \\ 
  & \\ 
 Male & 0.134$^{***}$ \\ 
  & (0.012) \\ 
  & \\ 
 Constant & 0.087$^{***}$ \\ 
  & (0.013) \\ 
  & \\ 
\hline \\[-1.8ex] 
Observations & 7,692 \\ 
R$^{2}$ & 0.018 \\ 
Adjusted R$^{2}$ & 0.017 \\ 
Residual Std. Error & 0.526 (df = 7687) \\ 
F Statistic & 34.624$^{***}$ (df = 4; 7687) \\ 
\hline 
\hline \\[-1.8ex] 
\textit{Note:}  & \multicolumn{1}{r}{$^{*}$p$<$0.1; $^{**}$p$<$0.05; $^{***}$p$<$0.01} \\ 
\end{tabular} 
\end{table} 


\end{document}
